\thispagestyle{plain}

\chapter*{Abstract}
Database Management Systems routinely use index structures to improve lookup performance compared to simple scans. Many recent index structures use the underlying data distribution to apply optimizations. What only a few do is also incorporate workload information. For this reason, we design two partitioning algorithms that utilize a workload sample to create meaningful partitions, which can be used as the leaf nodes of a customizable index. We evaluate the lookup times for different query workloads on such an index and compare it to state-of-the-art index structures. The two workload properties that we investigate are \textit{frequency} and \textit{query type} of individual data points. We found that when partitioning for query type purity, the resulting index can easily change its data structures according to the query type in one partition. Such changes can result in an improvement of the average lookup time from 350 ns to 250 ns. For the frequency partitioning, we found that without optimizing the index to incorporate frequency information, it is not competitive with the state-of-the-art. However, we also found that the lookup performance was comparable to a B-tree even in the worst case.