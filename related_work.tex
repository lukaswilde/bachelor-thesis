\thispagestyle{plain}
\chapter{Related Work}

In this chapter, I present work related to the topic of workload-based data partitioning.
Additionally, we reference work that deals with data partitioning in a broader sense.

\begin{enumerate}
\item \emph{Baseline indexes}
    \begin{itemize}
        \item B-tree \cite{Bayer1970-rh}
        \item ART index \cite{Leis2013}
    \end{itemize}
\item \emph{Hybrid/Adaptive Index Structures}

    \begin{itemize}
    \item Adaptive Hybrid Indexes \cite{Anneser2022} for context and compacting/decompressing criterias
    \item GENE \cite{Dittrich2021} for the approach to look at indexes as logical components and combining them, generic
    search briefly to iterate over starting options and give our partitioning as possible better starting point.
    \end{itemize}

\item \emph{Learned Index Structures}

    Emphasize how they use the underlying data distribution
    \begin{itemize}
    \item FITing-Tree \cite{Galakatos2019} as introduction to PGM (brief)
    \item PGM-index \cite{Ferragina:2020pgm} for optimal linear piecewise partitioning
    \item RMI/ALEX \cite{Kraska2018}, \cite{Ding2020} with their tree-like model structure, ALEX to improve upon caveats of RMI (updatability)
    \end{itemize}

\item \emph{Distributed (Database) Systems}

    Context: data partitioning in the sense that different partitions can be stored on different nodes of the distributed system
    \begin{itemize}
    \item Schism \cite{Curino2010} for their workload-centric approach to data partitioning
    \end{itemize}
\end{enumerate}
